%%%%%%%%%%%%%%%%%%%%%%%%%%%%%%%%%%%%%%%%%%%%%%%%%%%%%%%%%%%%%%%
\documentclass[12pt]{article}

%##################Packages##################################
%\usepackage[notref,notcite]{showkeys} %SHOWS LABELS

\usepackage[latin2]{inputenc}
\usepackage[usenames]{color}
\usepackage{latexsym}
\usepackage{epsf}
\usepackage{epsfig}
%
\usepackage{amssymb,amsmath}
\usepackage{mathrsfs}
\usepackage{empheq}
\usepackage{slashed}
\usepackage{listings}
\usepackage{url}
\usepackage[hidelinks]{hyperref}

\usepackage{graphicx}
\usepackage[font=footnotesize]{caption}
\usepackage{subcaption}
\usepackage{sidecap}
\usepackage{tikz}
\usetikzlibrary{shapes,arrows}
\usepackage{multicol}
\linespread{1.25}
\usepackage{algorithm}
\usepackage[noend]{algpseudocode}

\makeatletter
\def\BState{\State\hskip-\ALG@thistlm}
\makeatother




\def\ni{\noindent}
\def\nn{\nonumber \\}
\def\cC{{\cal C}}
\def\cP{{\cal P}}
\def\cR{{\cal R}}
\def\cL{{\cal L}}
\def\cT{{\cal T}}
\def\cW{{\cal W}}
\def\cM{{\cal M}}
\def\cN{{\cal N}}
\def\Z2{{\mathbb Z}_2}
\def\sgn{{\rm sgn}}
\def\pa{\partial}
\def\ud{\mathrm{d}}
\def\dis{\displaystyle}

\newlength{\dinwidth}
\newlength{\dinmargin}
\setlength{\dinwidth}{21.0cm}
\textheight22.7cm \textwidth17.7cm
\setlength{\dinmargin}{\dinwidth}
\addtolength{\dinmargin}{-\textwidth}
\setlength{\dinmargin}{0.5\dinmargin}
\oddsidemargin -2.25cm
\addtolength{\oddsidemargin}{\dinmargin}
\setlength{\evensidemargin}{\oddsidemargin}
\setlength{\marginparwidth}{0.9\dinmargin}
\marginparsep 8pt \marginparpush 5pt
\topmargin -57pt
\headheight 12pt
\headsep 30pt
\footskip 24pt



\newcommand{\argmax}{\operatornamewithlimits{argmax}}

%%%%%%%%%%%%%%%%%%%%%%%%%%%%%%%%%%%%%%%%%%%%%%%%%%%%%%%%%%%%%%%%
\begin{document}
\lstset{language=Lisp}


\thispagestyle{empty}

\vspace*{15mm}

\begin{center}
{\Large\bf
An Interpreter for a LISP-like Language
}

\vspace*{5mm}

\end{center}
\vspace*{5mm} \noindent
\vskip 0.5cm
\centerline{\bf
Lewis Smith
}
\centerline{
8933715
}

\vskip 5mm
\vskip 5mm
\centerline{\em University of Manchester }

\vskip 20mm
\vskip 20mm
\section*{Abstract}



%%TODO remove ~100 more words

This report details the design and implementation of a minimal
interpreter for a language in the LISP family, using C++. Encapsulation
in classes allows the separation of the main components of the
interpreter: the lexer, parser, symbol table, garbage collector, and
language data structures. The datatypes of the language are
implemented as a hierarchy deriving from an abstract base class, which
allows the polymorphic use of STL containers to store and manipulate
these types. Exceptions are used to record and recover from errors in
user-input commands.

 
%%%%%
%%%%%
\section{ Introduction }
%%%%%
%%%%%

An \textit{interpreter} is a program that directly executes a series
of commands, transforming instructions in the form of source code in a
specified language into computer actions. `Real world' interpreters
for production languages are generally programs of considerable size
and complexity, however if performance is not a priority, a minimal
interpreter can be implemented fairly concisely. Lisp was chosen as
the basis language for this interpreter\footnote{Technically speaking,
the language implemented is based on lisp rather than being a lisp: it
is compliant with neither the common lisp or scheme standards.} due to
the conceptual simplicity of it's syntax and semantics, which greatly
simplifies the implementation. All of lisps core syntax can be
expressed as follows:



\begin{lstlisting}
(function argument1 argument2 ...)
\end{lstlisting}


which translates to `apply function to arg1, arg2, arg3...'. All lisp
commands are \textit{expressions} rather than statements: that is,
they all evaluate to and return a value. All lisp objects are
\textit{first class}, meaning that they are all treated the same way,
and can be passed around, put inside data structures, and passed as
arguments to functions. This version of lisp
makes use of only one compound datatype, the heterogeneous
linked list, which is represented as a space separated list enclosed
by parentheses. Another feature of lisp that simplifies the interpreter
is that lisp programs are written as lists, evaluated as function calls
as desribed above. Therefore, the structure of a lisp program can be 
represented as a lisp datatype (a list of lisp objects), meaning there
is no need for a dedicated parse tree structure. This property is known
as \textit{homoiconicity}.

The interpreter works on an input string in stages:
\begin{enumerate}
	\item \textit{Lex} the input, that is, convert the
	representation of the program as a string of characters into a
	sequence of tokens with semantic meaning. 
	\item \textit{Parse}
	the output of the lexer, building a data structure that represents
	the program or expression to be executed. 
	\item Evaluate the parse tree,
	executing the instructions it represents.
\end{enumerate}


\section{Design}
\subsection{S-Expressions}
\label{section:sexp}

Lisp code consists of symbolic expressions or `S-Expressions'. An
s-expression is either a singular literal, an `atom' or identifier, or
a list of s-expressions. S expressions are implemented as an abstract
class, which the specialised types of the language all inherit from.
The most important method of the s expression base class is
\textit{eval}: this method takes a reference to an instance of the
environment class, which represents the current state of the
interpreter, and returns a pointer to an s-expression that represents
the value of the object when evaluated. The behaviour of this function
depends on the type: most types are their own values, and simply
return their own addresses. The exceptions are lists,
which are evaluated as function calls, returning a pointer to a new
s-expression holding the result of the call, and atoms, which return
the result of looking up that string in the symbol table (throwing an
exception if the atom is undefined).



The second virtual method allows a visitor pattern, which allows
another class, the \textit{visitor}, to dispatch methods based on the
actual type of the visited object at runtime. This is used to have
other classes that can act polymorphically on s-expressions, as in the
Representor classes that print representations of s-expressions to
strings. The same functionality could have been achieved using virtual
functions instead; however, this way is slightly neater because it
achieves allows adding or changing the behaviour of visitor classes
without having to modify the main class hierarchy. For example, the
implementation of the display representor class which behaves
differently on strings is trivial using the visitor pattern, but would
be cumbersome if new virtual functions had to be added to every
s-expression class.



All fields of s-expressions are made const where possible. Once a
value is created, it cannot be modified. Modification is only possible
by creating a slightly modified copy. This may seem inefficient:
however, since it allows pass by reference semantics, it actually
simplifies things. For example, consider the following small program


\begin{lstlisting}
	(define x '("a" "b" "c" "d" "e"))
	(define y (car x)) ;; y is "a" 
\end{lstlisting}

Since we know that the elements of the list are immutable, it is safe
to have two variables, x and y, that actually point to the same object
in memory ("a"). Using mutation behind the scenes could allow some
code to be more efficient: however, determining when it is safe to
mutate an object in user code is a difficult problem, and would
require an optimisation pass over the parse tree, possibly compiling
the parse tree to an intermediate bytecode before evaluating it. While
a common approach in production compilers, this was deemed beyond the
scope of this project. Using immutable objects is a simple way to ensure
the interpreter behaves correctly.

The use of a separate class to manage the memory of all s-expression
classes makes their definitions look somewhat unusual: in particular,
lists and user-defined functions all contain pointers to heap assigned
memory, but since no s-expression actually \textit{owns} memory, their
destructors and copy constructors are all trivial: it is safe to copy
an sexp pointer provided it was allocated correctly using the heap
class (see section \ref{section:garbage}). Combined with the
immutability of the underlying objects, this makes passing round the
raw pointers to s-expressions, which is done frequently,
much safer than it may first appear.


\subsection{Lexing and Parsing}
\label{section:parser}

Due to the simplicity of lisp's grammar, the lexer and parser are
fairly simple. Both expose a public interface that hides the details
of how the stream is actually processed, separating this from the rest
of the interpreter. One possible design choice that was not
implemented was to hide the parsing behind a stream extraction
operator for s expressions. A stream insertion operator was used to
hide the Representor class, so the absence of this is slightly
asymmetrical. However, I wanted the parser to throw exceptions, so
that the error handling mechanism could be uniform. This would break
the pattern for extraction operators, which conventionally signal
error by setting a failbit on the stream, and as a result I decided
that it would be less confusing to call the parser's read\_sexp
function explicitly than have an operator with unusual behaviour.

The lexer tracks its line number and position in the input file,
which are reported when the code encounters an error in order to aid
debugging.


\subsection{Environments}

The environment class implements a symbol table, a data structure that
maps defined symbols in the language to their values, which is used to
evaluate atoms. The symbol table is implemented using a hash table,
the STL unordered map. A hash table is a data structure that allows
indexing by an arbitrary type, such as strings, and provides average
constant time insertion, lookup and deletion, so is well suited to
this purpose compared with sequence-type structures like vectors or
lists, which are expensive to search. The environment class wraps the
allowed access to this table via the def and lookup functions. The
other functionality the base class provides is capture scope, which
returns a new environment with a symbol table that is a copy of the
current state. This allows the implementation of lexical closure, the
implicit capture of variables from an enclosing scope by a function 
definition.


The global environment is a specialisation of the env class. The most
notable difference is that it contains a heap, the class which is
responsible for garbage collection of the interpreted language. The
global env forwards calls to manage, which tracks the assignment of
new lisp objects, to the heap. Non-global envs must be created from
within a global env, and they retain a pointer to it's location in
order to forward manage calls to the global env. As a result of
this, it is very important that, once created, the global env is not
copied or even moved from its original location. Therefore, all of
it's move and copy constructors are explicitly deleted.

The global env's constructor also initialises the symbol table by
binding a modest set of language builtins so they are available to the
user. The builtin functions are defined in the file primitives.h and
primitives.cc.


\subsection{Garbage Collection}
\label{section:garbage}


Lisp is a garbage collected language, meaning that the responsibility
for manually managing heap memory is delegated to the language
runtime, rather than being the responsibility of the programmer as in
C++. As C++ does not have a garbage collector, it is necessary to
manually implement one for the lisp runtime. This must track the usage
of memory, detecting when a heap block is no longer needed by anything
and cleaning up such memory as needed. An object is in use by lisp if
it is reachable from the symbol table, directly or indirectly: if no
way to reach it exists then it is garbage.


\begin{figure}
  \centering
  \includegraphics[width=0.5\textwidth]{memory.png}
  \caption{\textit{
  A cartoon of the operation of mark-and-sweep. The table to the left of
  the figure represents the symbol table. Arrows between objects A and B
  denote the relationship ``A contains a pointer to B''. The red arrows
  represent the path taken by the marking phase: Starting at each name
  in the symbol table, all memory that is reachable from that name is
  traced out. Objects not reachable by these paths are no longer
  reachable by any part of the runtime, and are thus garbage that can be
  collected in the sweep phase.
  }}
\end{figure}

One option for a garbage-collection like facility that is provided in
C++ 11 is reference counting, where an pointer to a object maintains a
record of how many other pointers that are also pointing to the same
memory exist in the program, deallocating them when the reference
count reaches zero. This is implemented by C++11's shared pointer
class. Unfortunately, reference counting is not a complete form of
garbage collection. In particular, it is able to leak memory if
circular references are possible: that is, if object A contains a
pointer to object B, but object B also contains a pointer to object A,
then their reference counts will never hit zero even if it is not
possible to reach either A or B.

%%since you hae used immutable variables, cycles aren't actually possible...
%%maybe leave the garbage collection as it is for now. Or possibly we can refactor
%%out the heap class? 

The simplest garbage collection algorithm that is guaranteed to
collect reference cycles is \textit{mark and sweep}. This algorithm
has two stages: in the first, memory that is reachable is marked, by
starting from the roots of the memory graph and traversing through it,
marking all memory that is reached. After all memory is marked, all
memory in use is swept. Any memory that was not marked can be safely
freed. This requires all memory locations used by the interpreter
runtime to be stored somewhere, so that the sweep phase can clean them
up later.

The design I chose to implement this is a class that maintains a
record of all memory allocated in a lisp context, providing a method,
manage, that can be used to wrap all calls to new, storing the
pointers in an appropriate structure together with a a Boolean flag
for the mark phase.

\section{Results \& Conclusion}


The implemented language is minimal, but is powerful enough to express
useful computation. The functionality of the language is demonstrated
by two example programs: \textit{tests.lisp} demonstrates all the
implemented builtin functions in order, demonstrating that the
interpreter works as designed. \textit{erastothenes.lisp} uses the
Sieve of Erastothenes algorithm to find all the prime numbers less
than 1000, demonstrating the use of the provided language to express a
non-trivial program. The garbage collector works, as validated by the
use of the memory tracking program \textit{valgrind} on Linux and by
the memory debugger in visual studio.


\begin{figure}
  \centering
  \includegraphics[width=\textwidth]{valgrind.png}
  \caption{\textit{
  Results from testing the garbage collection using the memory tracking
  program valgrind under Linux (compiled with clang). The small amount 
  of memory left unfreed is not allocated by my program, but by some import:
  some header files allocate memory for their own internal use, which is 
  not freed explicitly since all memory is freed on program exit in any case.
  }}
\end{figure}



There are a few notable omissions from my version of `lisp': the first
is that I only use a single numeric type, a double precision floating
point. It would be convenient to have a dedicated integer type, as in
most programming languages. However, floating points are good enough
for most use cases, and having only a single numerical type simplfies
the parser and means the numerical functions need not handle casts
between different numeric types. A minority of mainstream programming
languages also make this trade off, the most prominent being
JavaScript.


Another omission is my implementation of lists and the builtin cons
function: in a traditional lisp cons does not need a list as the
second argument, but can take anything. A cons onto a non-list object
is called a pair or an improper list, and is written as follows:

\begin{lstlisting}

(a . b)

\end{lstlisting}

I did not implement this because it complicates the implementation of
the list type and is not particularly useful. Clojure, a modern lisp,
also leaves out this feature.


Some of the more advanced features often found in lisp-family
languages are also left out; user defined functions with variable
numbers of arguments, and lexical macros. If the project was extended
these would be obvious extensions. Another would be providing builtins
for manipulating strings: while they exist in the language, at the
moment they cannot really be manipulated in any meaningful way.


There are a few areas of my implementation where clarity and
simplicity was chosen over performance: this was deemed acceptable due
to the scope of the project, but would possibly require changing if
the implementation was extended. The most notable of these are the
garbage collector and the parser: both use explicitly recursive
strategies, which lead to clear code, but mean that adverserial input\footnote{
For example, a very long file consisting of nothing but open brackets.
}
could cause the program to overflow the stack and crash. This could 
be avoided by re-writing the parser class to use a heap allocated buffer
and avoid explicit recursion. In practice this is unlikely to cause any
problems for reasonable input, so I decided to stick to the simpler version.
Due to the class design, changning the parsing strategy could be done without 
modifying the rest of the code.

Overall, however, I feel that this project is a successful
demonstration of effective use of classes and the principles behind a
working interpreter.

\end{document}
